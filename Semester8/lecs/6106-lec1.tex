\chapter{Statistics of Binary Classification}

Let $\mathcal{X}$ be an input space. Consider the following statistical model for labelling elements of $\mathcal X$: an underlying unknown distribution $\mathcal{D}$ over $\mathcal{X} \times \{0, 1\}$. We are given a \spl{training set} $S = \{(x_1, y_1), \ldots, (x_n, y_n)\}$ drawn iid from $\mathcal D$ and wish to ``learn'' a \spl{classifier} $g : \mathcal{X} \to \{0, 1\}$ that approximates the distribution $\mathcal{D}$ as best as possible. Consider the following loss of a classifier $g$ on a sample $(x,y)$:
\[
    L(g, (x, y)) = \begin{cases}
        1 & \text{if } g(x) \neq y, \\
        0 & \text{otherwise}.
    \end{cases}
\]

The average loss, then, is given by
\[
    L(g) := \ee{(x,y)\sim\mathcal D}{L(g, (x,y))} = \pp{(x,y)\sim\mathcal D}{g(x) \neq y}.
\]

We only have a finite sample to measure losses of potential classifiers: we define the \spl{empirical} loss of a classifier on the sample by
\[
    L_n(g) := \frac{1}{n} \sum_{i=1}^n L(g, (x_i, y_i)). 
\]
Notice that it is a random variable since it depends on the random sample $S$.

Suppose that an algorithm given sample $S$ returns a classifier $\widehat{g}_n$. We are interested in two questions about the classifier.

\begin{enumerate}
    \item Is $L(\widehat{g}_n)$ close to $\displaystyle{\inf_{g\in\mathcal C}L(g)}$? This is the \spl{accuracy} question.
    \item Is $L(\widehat{g}_n)$ close to $L_n(\widehat{g}_n)$? This is the \spl{generalization} question.
\end{enumerate}

Consider the natural algorithm for binary classification in Figure \ref{alg:min-emp-loss} that simply chooses the classifier minimizing \emph{empirical loss} (what more do we have to go on?).

% algorithm for minimum empirical loss
\begin{figure}[!h]
    \label{alg:min-emp-loss}
    \begin{algorithm}[H]
        \begin{algorithmic}[1]
            \State \textbf{Input:} Sample $S = \{(x_1, y_1), \ldots, (x_n, y_n)\}$.
            \State \textbf{Output:} Classifier $\widehat{g}_n$.
            \State $\widehat{g}_n = \arg\min_{g\in\mathcal C} L_n(g)$.
            \State \textbf{return} $\widehat{g}_n$.
        \end{algorithmic}
    \end{algorithm}    
    \caption{Just minimize Empirical Loss!}
\end{figure}

Let's estimate its accuracy. Assume that $\displaystyle{g^* = \arg\min_{g\in\mathcal C}L(g)}$ is the best classifier in the class $\mathcal C$ (called the \spl{Naïve Bayes classifier}). We have

\begin{align*}
    L(\widehat{g}_n) - L(g^*) &= L(\widehat{g}_n) - L_n(\widehat{g}_n) + L_n(\widehat{g}_n) - L(g^*)\\
    &\leq L(\widehat{g}_n) - L_n(\widehat{g}_n) + L_n(g^*) - L(g^*)\\
    &\leq 2\sup_{g\in \mathcal C}|L_n(g) - L(g)|.
\end{align*}

The goal now is to boudn the quantity $\sup_{g\in \mathcal C}|L_n(g) - L(g)|$, as $n$ grows large. If it can be shown to go to $0$, we have asymptotically perfect accuracy. Notice that for any particular $g$, the value $L_n(g)$ is a random variable with expectation (over $S$) being $L(g)$, i.e. $\ee{S}{L_n(g)} = L(g)$. As $n$ grows large, one expects that $L_n(g)$ is close to $L(g)$. However, do we have a uniform upper bound for all classifiers in $\mathcal C$? If $\mathcal C$ is finite, this is easy to see; however, in the infinite case, it is not clear. We essentially need a \emph{uniform} LLN.

\textbf{Remark}. For those familiar with some analysis, the situation here is similar to uniform convergence of a sequence of functions; we are looking to check if the sequence of functions $\{L_n\}$ converges uniformly to $L$.

\section{Uniform Law of Large Numbers}

Suppose that $S = \{X_1, \cdots, X_n\}$ is a list of iid random objects (distributed according to $\mathcal D$, say) taking values in $\mathcal X$. Let $\mathcal F$ be a class of real-valued functions $\mathcal X\to\mathbb R$. What can we say about the random variable $Z$ over $\mathcal D^S$ defined by
\[
    Z = \sup_{f\in\mathcal F} \left| \frac{1}{n} \sum_{i=1}^n f(X_i) - \ee{\mathcal D}{f(X)} \right|?
\]

[Of course, this only makes sense when $\ee{X}{f(X)}$ is well-defined for all $f\in\mathcal F$.]

In particular, we are interested in the following three questions.

\begin{enumerate}
    \item Whether $Z \to 0$ when $n\to\infty$ (asymptotic question).
    \item Can we obtain non-asymptotic guarantees, at large $n$?
    \item Can we provide conditions such that $Z$ converges to $0$?
\end{enumerate}

We have already seen that the ULLN can analyse binary classification. Another application is in $M$-estimation.

\subsection{$M$-estimation (intro)}

Many problems in statistics are concerned with estimators of the form 

\[
    \widehat{\theta}_n = \arg\max_{\theta\in\Theta} \frac{1}{n}\sum_{i=1}^n m_{\theta}(x_i),
\]
where $X_1, \cdots, X_n$ are iid observations. Here, $\Theta$ is the parameter space, and $m_{\theta} : \mathcal X \to \mathbb R$ is a real-valued function parameterized by $\theta$.

\begin{example}
    [familiar $M$-estimators]

    \noindent
    \begin{enumerate}
        \item $m_\theta(x) = \log p_\theta(x)$ where $p$ is a family of distributions parameterized by $\theta$. This is the \spl{maximum likelihood estimator}, i.e. $\widehat{\theta}_n$ is the MLE estimate.
        \item $m_\theta(x) = -(x - \theta)^2$. This estimator is the \spl{sample mean}, i.e. $\widehat{\theta}_n = \frac{1}{n}\sum_{i=1}^n x_i$.
        \item $m_\theta(x) = -|x - \theta|$. This estimator is the \spl{sample median}.
    \end{enumerate}
\end{example}

In $M$-estimation, the target quantity for $\widehat{\theta}_n$ (what we want $\widehat{\theta}_n$ to approach) is 
\[
\theta^* = \arg\max_{\theta\in\Theta} \e{m_{\theta}(X)}.
\]

Similar to binary classification accuracy, we want $d(\widehat{\theta}_n, \theta^*) = |\widehat{\theta}_n - \theta^*|$ to be small. It turns out (we will see this later) that
\[
    d(\widehat{\theta}_n, \theta^*) \leq 2\sup_{\theta\in\Theta} \left| \frac{1}{n}\sum_{i=1}^n m_{\theta}(X_i) - \e{m_{\theta}(X)} \right|,
\]
which is another instance of the use of the uniform LLN.

\subsection{Statement and Proof}

Back to the uniform LLN. To show some results, we first need the following observation.

\b{Key observation}. $Z$ concentrates around \[\displaystyle{\ee{S}{Z} = \ee{S}{\sup_{f\in\mathcal F} \left| \frac{1}{n} \sum_{i=1}^n f(X_i) - \ee{\mathcal D}{f(X)} \right|}}.\]

We can then control $\e{Z}$ through techniques like \spl{symmetrization} (leading to Rademacher complexity) and \spl{chaining} (leading to VC dimension).

\textbf{Remark}. A family of functions $\mathcal F$ is called \spl{Glivenko-Cantelli} if $Z \to 0$ a.s. as $n \to \infty$ (more on this later).